\documentclass[12pt, a4paper]{article}
\usepackage [T1]{fontenc}
\usepackage[margin=2.5cm]{geometry}
\usepackage{graphicx}
\usepackage[utf8]{inputenc}
\usepackage[english]{babel}
\usepackage{bm}
\usepackage{float}
\usepackage{pifont}

\usepackage[toc,page]{appendix}
\usepackage[nottoc]{tocbibind}
\setcounter{tocdepth}{1}


\usepackage{algorithmic}

\usepackage[normalem]{ulem}
\usepackage{multirow}
\usepackage{tcolorbox}
\usepackage{adjustbox}
\usepackage{array}
\usepackage{tabularx}
\usepackage{longtable}
\usepackage{tabulary}
\usepackage{color}
\usepackage{colortbl}
\usepackage{gensymb}
\usepackage{pdfpages}
\usepackage{caption}
\usepackage{subcaption}
\usepackage{diagbox}
\usepackage{hyperref}
\usepackage[section]{placeins}
\usepackage{bbm}
\usepackage[linesnumbered,algoruled]{algorithm2e}

%% Biblography
\usepackage[style=numeric-comp,
            hyperref=auto,
            backend=biber,
            backend=bibtex]{biblatex}


% Code format
\usepackage{listings}
\usepackage{minted}
\setminted{
	frame=lines,
	framesep=2mm,
	baselinestretch=1.2,
	linenos,
	fontsize=\footnotesize,
	breaklines=true
}

% Math ops
\usepackage{mathtools}
\usepackage{amsmath, amssymb}
\DeclareMathOperator*{\argmax}{arg\,max}
\DeclareMathOperator*{\argmin}{arg\,min}
\DeclareMathOperator*{\EX}{\mathbb{E}}% expected value

\DeclarePairedDelimiter{\ceil}{\lceil}{\rceil}
\DeclarePairedDelimiter{\floor}{\lfloor}{\rfloor}

\newcommand{\dunderline}[1]{\underline{\underline{#1}}}
\newcommand{\euler}[1]{\ensuremath{\ \mathrm{e}^{#1}}}
\newcommand{\ma}[1]{\textbf{#1}}
\newcommand{\ve}[1]{\textbf{#1}}
\newcommand{\ap}[1]{\mathrm{\overline{#1}}}
\newcommand{\pp}[1]{\left( {#1} \right)}
\newcommand{\abs}[1]{\left|   {#1} \right|  }
\newcommand{\forn}[2]{{#1} \rightarrow {#2} \quad \text{for} \quad n \rightarrow \infty }
\newcommand{\forN}[2]{{#1} \rightarrow {#2} \quad \text{for} \quad N \rightarrow \infty }
\newcommand{\folg}[3]{ \{#1_{#3}\}_{#3 = #2}^{\infty} }
\newcommand{\folgc}[3]{ \{#1_{#3}\}_{#3 = #2}^{c} }
\newcommand{\ip}[2]{\langle{#1} , {#2}\rangle}

\newcommand{\RR}{ \mathbb{R} }
\newcommand{\NN}{ \mathbb{N} }
\newcommand{\CC}{ \mathbb{C} }
\newcommand{\for}{  \quad \text{for} \quad  }
\newcommand{\as}{  \quad \text{as} \quad  }

\usepackage{hyperref}
\addbibresource{main.bib}
\author{Minh}
\begin{document}
	\begin{titlepage}
		\centering
		\includegraphics[width=\textwidth]{KuLogo.png}
		\par\vspace{1cm}
		\vspace{1cm}

		{\scshape\Large Automatic Differentiation \\
			\par}

		\vspace{1.5cm}
		{\Large\itshape  Minh Duc Tran }\\
		\vspace{0.5cm}
		{\scshape\large cwz688 \\ \par}
		\vfill

		{\Large\scshape Master project \par}

		\vfill
		\par
		\vfill
		{\large \today\par}
	\end{titlepage}
	\newpage

\begin{abstract}
	Automatic diff is good
\end{abstract}
\section{Introduction}
Many important problems require taking derivatives to be solved. Such problems include 
Bundle Adjustment (BA) (cite) and XX. While finding the derivative for simple trivial functions 
can be done by hand. This method is accurate and often the 
fastest method when executed but it scales poorly for problems of real world relevance as 
finding the derivatives by hand is time consuming and can be error prone.  We hence seek to 
automate the process of finding these derivatives. \newline 

The simplest method to implement is \textit{numerical differentiation} methods 
but is slow in execution as it requires two evaluations per scalar derivative. 
It also suffers from numerical imprecision. A more precise method is 
\textit{symbolic differentiation}, which are often used in  ''mathematical'' languages
like Matlab, Mathematica or on your old TI-89.  This generates exact symbolic derivates, but 
is memory intensive and slow to compute. It can also not handle 
complex logic such as unbounded loops and hence is not appropriate for general computer programs. 
For solving most of these issues we turn to \textit{automatic differentiation}, which can 
compute exact derivatives for an arbitrary computer program. The caveat is however 
that the implementation must be carefully thought out. We will in this report
see how one can implement automatic differentiation in a way that minimizes 
the computational ....



\section{Automatic differentiation}
Automatic differentiation relies on the fact every computer program, regardless of how complex it is, 
is a composition of arithmetic operations, like addition, multiplication etc. and exploits the chain rule 
for composing these derivatives. For example consider the composition of $n$ scalar functions:
$f(x) = f^n \circ f^{n-1} \circ \cdots \circ f^1 \circ x$. The chain then simply states that the derivate of 
$f$ w.r.t. $x$ is 
\begin{equation}
\frac{\partial f}{\partial x} = \frac{f^L}{f^{L-1}} \frac{f^{L-1}}{f^{L-2}} \cdots \frac{f^1}{x}
\label{eq:f}
\end{equation}
By formulating our function $f(x)$ as a program we are able to see how we can 
use AD to compute $\frac{\partial f}{\partial x}$. AD has two modes, forward- and reverse. Both of which 
both have it's own advantages and disadvantages.  
The main difference between these two is in which direction the terms of each partial derivatives
are computed. For example in Eq. \ref{eq:f} forward-mode starts from the right most term, i.e. it first computes $\frac{f^1}{x}$
and traverses to the left. On the other hand does reverse-mode start from $\frac{f^L}{f^{L-1}}$ and accumulate to the right. \newline 
The example above is simplified for most AD problems and we in the general case 
have a multi-dimensional function $f : \mathbb{R}^n \to \mathbb{R}^m$  which would then have the $m\times n$  Jacobian matrix $J_f$:
\begin{figure}[H]
	$$ J_{f} = \left(\begin{matrix}
	\frac{\partial f_1}{\partial x_1} & \cdots & \frac{\partial f_1}{\partial x_n} \\
    \vdots & \ddots & \vdots \\
	\frac{\partial f_m}{\partial x_1} & \cdots  &  \frac{\partial f_3}{\partial x_n}\\
	\end{matrix}\right) $$
	\caption{$m\times n$ Jacobian matrix of $f : \mathbb{R}^n \to \mathbb{R}^m$}
\end{figure}
Given an input vector $x \in \mathbf{R}^n$  forward-mode evaluates $f$ once and computes a column of $J_f$. Hence to compute 
the entire Jacobian we need to perform $n$ sweeps or said differently we need to evaluate $f$ $n$ times. 
Reverse-mode works in slightly differently. Given an output space vector $y \in \mathbf{R}^m$, a sweep will compute a 
row of $J_f$. Hence do we need to perform 1 application of $f$ and $m$ reverse-mode sweeps.  We can now see the 
computational difference between the two modes. When $n << m$ forward-mode is more efficient and vice versa for $n >> m$. 
Applications such neural networks usually have large input dimensions $n$ compared to output $m$. 
Libraries like Tensorflow are reverse-mode based implementations. One downside to reverse-mode is that 
it requires access to intermediate variables and hence require more memory. 

\subsection{Forward-mode implementation}
The implementation of forward-mode can be done with two strategies:
\subsubsection*{Source code transformation}
One of the oldest methods for AD is source code transformation. Here an AD tool, (or preprocessor), rewrites our function that we wish to 
compute the derivative of. A preprocessor then applies the differentiation rules as prescribed by the elementary operators and chain rules, 
and returns the augmented source function which now includes statements for computing the derivatives. The resulting source
code can be compiled and executed as usual. There are few examples of such implementations. 
For example has Tangent\cite{DBLP:journals/corr/abs-1711-02712} implemented AD support for a subset of the Python language, 
while Tapenade has AD tools for transforming C and Fortran functions. The major downsides of source code transformation is that 
they can only use information available at compile time and developing the tools is non-trivial and requires a lot of effort. 
	
\subsubsection*{Operator overloading}
Operator overloading (OO) is far more common and simpler to implement.
Forward-mode OO is usually implemented using \emph{dual numbers}, which extends the real numbers 
of the programming language by a differential component. The real number operators are then lifted to
operate on the dual number. Hence can implementations usually be seen as an embedded DSL in form of a library within a 
more general host language. These DSL can then leverage the constructs and features of the host and can be imported
as any other library. This ensures a flexible and portable advantage to embedded DSL. Examples of such DSL libraries are 
\texttt{Julia ForwardDiff}\cite{RevelsLubinPapamarkou2016} and \texttt{Stan}\cite{DBLP:journals/corr/CarpenterHBLLB15}. 



\subsubsection{Example}
\subsubsection{Problems}
\subsubsection{Sparsity}
This section is based on Julia's approach to sparse AD computation. 
Link: \url{https://github.com/JuliaDiffEq/SparseDiffTools.jl}. 


\subsection{Example: Data-independent sparsity}

Below is a function which performs a 1D-convolution on the input array $x$ of dimension $n$. 
The Jacobian of such a function is the tri-diagonal matrix shown in Figure \ref{fig:tridiag}. 
\begin{minted}{C}
int* f (int* x, int n) {
retval = malloc(sizeof(int) * n);
for (int i = 1; i < n - 1; i++) {
retval[i] = x[i-1] + x[i] + x[i+1];
}
retval[0] = x[0] + x[1];
retval[n-1] = x[n-1] + x[n-2];
return retval;
}
\end{minted}
\begin{figure}[H]
	$$ J_{f} = \left(\begin{matrix}
	\frac{\partial f_1}{\partial x_1} & \frac{\partial f_1}{\partial x_2} & 0 & 0 & 0 \\
	\frac{\partial f_2}{\partial x_1}& \frac{\partial f_2}{\partial x_2} & \frac{\partial f_2}{\partial x_3} & 0 & 0\\
	0 & \frac{\partial f_3}{\partial x_2} & \frac{\partial f_3}{\partial x_3} & \frac{\partial f_3}{\partial x_4} & 0\\
	0 & 0 & \frac{\partial f_4}{\partial x_3} & \frac{\partial f_4}{\partial x_4} & \frac{\partial f_4}{\partial x_5} \\
	0 & 0 & 0 & \frac{\partial f_5}{\partial x_4} & \frac{\partial f_5}{\partial x_5}
	\end{matrix}\right) $$
	\caption{Jacobian from \texttt{f} for $n=5$}
	\label{fig:tridiag}
\end{figure}
The sparsity of the Jacobian in this case is independent of the data, e.g. $x_1$ will always be independent of $f_5$ 
or more generally the off tri-diagonal elements will always be zero. 
Of course the case where the input value is zero, e.g. $x_1 = 0$  the entry $\frac{\partial f_1}{\partial x_1}$ will also be zero, 
which would then be \emph{data} dependent.

\subsection{Example: Bundle Adjustment - Data-dependent}
For a data-dependent example we use  Bundle Adjustment \footnote{Link to Wiki} as example and 
show that the layout of the Jacobian depends on how we represent our data as shown in Figure \ref{fig:jacobian-matrix-for-a-bundle-adjustment-problem-con-sisting-of-3-cameras-and-4-points}.
Note that for showing the natural sparsity of the Jacobian we have to represent the input data as one long vector
i.e. our input data has the form $\lbrack \texttt{cam1} \ \texttt{cam2} \ \texttt{cam3}\ X_1\ X_2\ X_3\ X_4 \rbrack $, 
where only one pair of \texttt{camX} and $X_i$ are non-zero for each data-row. 
The Jacobian sparsity pattern hence depends on the input, for example if we swap the columns \texttt{cam1} and \texttt{cam2}, keeping all else the same,
the sparsity pattern of the Jacobian also changes. 
\begin{figure}[H]
	\centering
	%\includegraphics[width=0.7\linewidth]{../../../Desktop/Jacobian-matrix-for-a-bundle-adjustment-problem-con-sisting-of-3-cameras-and-4-points}
	\caption{Data dependent sparse matrix stemming from Bundle Adjustment (Stolen from  \url{https://www.researchgate.net/figure/Jacobian-matrix-for-a-bundle-adjustment-problem-con-sisting-of-3-cameras-and-4-points_fig1_284154527})}
	\label{fig:jacobian-matrix-for-a-bundle-adjustment-problem-con-sisting-of-3-cameras-and-4-points}
\end{figure}


\subsection{Sparsity w.r.t. performance}
For Forward-mode AD we can reduce the computation of both types of sparsity with a \texttt{colour} vector, 
which  denotes which input parameters are independent (or dependent) of each other. 
This corresponds to solving a graph-colouring problem.
For example for the function \texttt{f} above we can use a color vector 
with three colours as each output value depends on at most three input values. 
A colour vector for $n=5$ could then be $\lbrack c_0, c_1, c_2, c_1, c_0\rbrack$ where input values with the 
same colour can be computed during the same forward-sweep, so we reduce the number of 
forward calls to the number of colours used. Worth noting is that \texttt{f}'s colour vector 
\emph{always}  contains 3 colours; independent of $n$ so the number of forward calls 
is effectively reduced by  $n-3 + 1$ in this case. (Plus one since we need to compute the colour vector)\newline 
For finding the colour-vector we can perform a forward sweep of the function 
with some random input and use some underlying data-structure to solve the graph-colouring problem. 
Alternatively can one let the user provide the color vector. 

\subsubsection{Bundle Adjustment revisited}
For Bundle Adjustment we can also use the same approach by seeding the forward sweep 
with the camera and point data. Using the data it's possible to figure out the dependencies and generate a 
coloring, which in this case would be a matrix, one for each data-row. 
In stochastic gradient descent methods, like used Bundle Adjustment, this color vector can provide a speed-up on subsequent 
computations of the Jacobian. 
However if the data is only used once, this approach requires that we compute the coloring again and again
and hence would not be very efficient. 



\subsection{Sparsity w.r.t. Memory}
Julia and STAN uses a sparse data structures but to my knowledge not feasible in Futhark?
Need more time to look into this. 


\section{Implementations}
\subsection{Optimizations}
\subsection{Motivation}


%
%  \section{Overview}
%  \begin{table}[H]
%    \centering
%    \begin{tabular}{l|lll}
%      \textbf{Langauge}  & Name & Mode & Implementation  \\ \hline
%      Julia                & JuliaDiff\footnote{https://www.juliadiff.org/} &  F/R  & OO \\
%      Swift (TensorFlow?)  & &     & \\
%      Tensorflow           & Tensorflow   & R  & OO \\
%      STAN                 &   &  & OO \\
%    \end{tabular}
%    \caption{Overview of languages}
%    \label{tab:overview}
%  \end{table}
%
%\subsection{Julia}
%In forwardDiff it uses extended dual numbers for higher order dimensions 
%
%
%
%\subsection{Tensorflow}
%
%Reverse-mode using gradient tape




\newpage
\nocite{*}
\printbibliography


\end{document}