\documentclass[12pt, a4paper]{article}
\usepackage [T1]{fontenc}
\usepackage[margin=2.5cm]{geometry}
\usepackage{graphicx}
\usepackage[utf8]{inputenc}
\usepackage[english]{babel}
\usepackage{bm}
\usepackage{float}
\usepackage{pifont}

\usepackage[toc,page]{appendix}
\usepackage[nottoc]{tocbibind}
\setcounter{tocdepth}{1}


\usepackage{algorithmic}

\usepackage[normalem]{ulem}
\usepackage{multirow}
\usepackage{tcolorbox}
\usepackage{adjustbox}
\usepackage{array}
\usepackage{tabularx}
\usepackage{longtable}
\usepackage{tabulary}
\usepackage{color}
\usepackage{colortbl}
\usepackage{gensymb}
\usepackage{pdfpages}
\usepackage{caption}
\usepackage{subcaption}
\usepackage{diagbox}
\usepackage{hyperref}
\usepackage[section]{placeins}
\usepackage{bbm}
\usepackage[linesnumbered,algoruled]{algorithm2e}

%% Biblography
\usepackage[style=numeric-comp,
            hyperref=auto,
            backend=biber,
            backend=bibtex]{biblatex}


% Code format
\usepackage{listings}
\usepackage{minted}
\setminted{
	frame=lines,
	framesep=2mm,
	baselinestretch=1.2,
	linenos,
	fontsize=\footnotesize,
	breaklines=true
}

% Math ops
\usepackage{mathtools}
\usepackage{amsmath, amssymb}
\DeclareMathOperator*{\argmax}{arg\,max}
\DeclareMathOperator*{\argmin}{arg\,min}
\DeclareMathOperator*{\EX}{\mathbb{E}}% expected value

\DeclarePairedDelimiter{\ceil}{\lceil}{\rceil}
\DeclarePairedDelimiter{\floor}{\lfloor}{\rfloor}

\newcommand{\dunderline}[1]{\underline{\underline{#1}}}
\newcommand{\euler}[1]{\ensuremath{\ \mathrm{e}^{#1}}}
\newcommand{\ma}[1]{\textbf{#1}}
\newcommand{\ve}[1]{\textbf{#1}}
\newcommand{\ap}[1]{\mathrm{\overline{#1}}}
\newcommand{\pp}[1]{\left( {#1} \right)}
\newcommand{\abs}[1]{\left|   {#1} \right|  }
\newcommand{\forn}[2]{{#1} \rightarrow {#2} \quad \text{for} \quad n \rightarrow \infty }
\newcommand{\forN}[2]{{#1} \rightarrow {#2} \quad \text{for} \quad N \rightarrow \infty }
\newcommand{\folg}[3]{ \{#1_{#3}\}_{#3 = #2}^{\infty} }
\newcommand{\folgc}[3]{ \{#1_{#3}\}_{#3 = #2}^{c} }
\newcommand{\ip}[2]{\langle{#1} , {#2}\rangle}

\newcommand{\RR}{ \mathbb{R} }
\newcommand{\NN}{ \mathbb{N} }
\newcommand{\CC}{ \mathbb{C} }
\newcommand{\for}{  \quad \text{for} \quad  }
\newcommand{\as}{  \quad \text{as} \quad  }

\addbibresource{main.bib}
\author{Minh}
\begin{document}
	\begin{titlepage}
		\centering
		\includegraphics[width=\textwidth]{KuLogo.png}
		\par\vspace{1cm}
		\vspace{1cm}
		{\scshape\Large Automatic Differentation \\
			NDAK15014U\par}
		\vspace{1.5cm}
		{\Large\itshape Duc Minh Tran }\\
		\vspace{0.5cm}
		{\scshape\large cwz688 \\ \par}
		\vfill
		{\Large\scshape Master project \par}
		\vfill
		\par
		\vfill
		{\large \today\par}
	\end{titlepage}
	\newpage
\begin{abstract}
	Automatic diff is good
\end{abstract}
\section{Introduction}
Many important problems require taking derivatives to be solved. Such problems include 
Bundle Adjustment (BA) (cite) and XX. While finding the derivative for simple trivial functions 
can be done by hand. This method is accurate and often the 
fastest method when executed but it scales poorly for problems of real world relevance as 
finding the derivatives by hand is time consuming and can be error prone.  We hence seek to 
automate the process of finding these derivatives. \newline 

The simplest method to implement is \textit{numerical differentiation} methods 
but is slow in execution as it requires two evaluations per scalar derivative. 
It also suffers from numerical imprecision. A more precise method is 
\textit{symbolic differentiation}, which are often used in  ''mathematical'' languages
like Matlab, Mathematica or on your old TI-89.  This generates exact symbolic derivates, but 
is memory intensive and slow to compute. It can also not handle 
complex logic such as unbounded loops and hence is not appropriate for general computer programs. 
For solving most of these issues we turn to \textit{automatic differentiation}, which can 
compute exact derivatives for an arbitrary computer program. The caveat is however 
that the implementation must be carefully thought out. We will in this report
see how one can implement automatic differentiation in a way that minimizes 
the computational ....



\section{Automatic differentiation}
Automatic differentiation relies on the fact every computer program, regardless of how complex it is, 
is a sequence of arithmetic operations, like addition, multiplication etc. There exist 
two AD modes, which are forward- and reverse-mode. We'll mainly focus on forward-mode in this report. 

\subsection{Forward-mode}
Introduction to derivatives
\subsubsection{Example}
\subsubsection{Problems}

While we'll focus on forward-mode AD in this report it's worth noting 
that both reverse-mode and forward-mode has it's use cases. In algorithms such
as backpropagation is the clear choice reverse-mode AD due to it's superior
efficiency. 

\section{Implementations}
\subsection{Optimizations}
\subsection{Motivation}


  \section{Overview}
  \begin{table}[H]
    \centering
    \begin{tabular}{l|lll}
      \textbf{Langauge}  & Name & Mode & Implementation  \\ \hline
      Julia                & JuliaDiff\footnote{https://www.juliadiff.org/} &  F/R  & OO \\
      Swift (TensorFlow?)  & &     & \\
      Tensorflow           & Tensorflow   & R  & OO \\
      STAN                 &   &  & OO \\
    \end{tabular}
    \caption{Overview of languages}
    \label{tab:overview}
  \end{table}

\subsection{Julia}
In forwardDiff it uses extended dual numbers for higher order dimensions 



\subsection{Tensorflow}

Reverse-mode using gradient tape





  \newpage
  \nocite{*}
  \printbibliography


\end{document}