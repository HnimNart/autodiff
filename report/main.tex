\documentclass[12pt, a4paper]{article}
\usepackage [T1]{fontenc}
\usepackage[margin=2.5cm]{geometry}
\usepackage{graphicx}
\usepackage[utf8]{inputenc}
\usepackage[english]{babel}
\usepackage{bm}
\usepackage{float}
\usepackage{pifont}

\usepackage[toc,page]{appendix}
\usepackage[nottoc]{tocbibind}
\setcounter{tocdepth}{1}


\usepackage{algorithmic}

\usepackage[normalem]{ulem}
\usepackage{multirow}
\usepackage{tcolorbox}
\usepackage{adjustbox}
\usepackage{array}
\usepackage{tabularx}
\usepackage{longtable}
\usepackage{tabulary}
\usepackage{color}
\usepackage{colortbl}
\usepackage{gensymb}
\usepackage{pdfpages}
\usepackage{caption}
\usepackage{subcaption}
\usepackage{diagbox}
\usepackage{hyperref}
\usepackage[section]{placeins}
\usepackage{bbm}
\usepackage[linesnumbered,algoruled]{algorithm2e}

%% Biblography
\usepackage[style=numeric-comp,
            hyperref=auto,
            backend=biber,
            backend=bibtex]{biblatex}


% Code format
\usepackage{listings}
\usepackage{minted}
\setminted{
	frame=lines,
	framesep=2mm,
	baselinestretch=1.2,
	linenos,
	fontsize=\footnotesize,
	breaklines=true
}

% Math ops
\usepackage{mathtools}
\usepackage{amsmath, amssymb}
\DeclareMathOperator*{\argmax}{arg\,max}
\DeclareMathOperator*{\argmin}{arg\,min}
\DeclareMathOperator*{\EX}{\mathbb{E}}% expected value

\DeclarePairedDelimiter{\ceil}{\lceil}{\rceil}
\DeclarePairedDelimiter{\floor}{\lfloor}{\rfloor}

\newcommand{\dunderline}[1]{\underline{\underline{#1}}}
\newcommand{\euler}[1]{\ensuremath{\ \mathrm{e}^{#1}}}
\newcommand{\ma}[1]{\textbf{#1}}
\newcommand{\ve}[1]{\textbf{#1}}
\newcommand{\ap}[1]{\mathrm{\overline{#1}}}
\newcommand{\pp}[1]{\left( {#1} \right)}
\newcommand{\abs}[1]{\left|   {#1} \right|  }
\newcommand{\forn}[2]{{#1} \rightarrow {#2} \quad \text{for} \quad n \rightarrow \infty }
\newcommand{\forN}[2]{{#1} \rightarrow {#2} \quad \text{for} \quad N \rightarrow \infty }
\newcommand{\folg}[3]{ \{#1_{#3}\}_{#3 = #2}^{\infty} }
\newcommand{\folgc}[3]{ \{#1_{#3}\}_{#3 = #2}^{c} }
\newcommand{\ip}[2]{\langle{#1} , {#2}\rangle}

\newcommand{\RR}{ \mathbb{R} }
\newcommand{\NN}{ \mathbb{N} }
\newcommand{\CC}{ \mathbb{C} }
\newcommand{\for}{  \quad \text{for} \quad  }
\newcommand{\as}{  \quad \text{as} \quad  }

\usepackage{hyperref}
\addbibresource{main.bib}
\author{Minh}
\begin{document}
	\begin{titlepage}
		\centering
		\includegraphics[width=\textwidth]{KuLogo.png}
		\par\vspace{1cm}
		\vspace{1cm}
		{\scshape\Large Automatic Differentiation \\
			\par}
		\vspace{1.5cm}
		{\Large\itshape  Minh Duc Tran }\\
		\vspace{0.5cm}
		{\scshape\large cwz688 \\ \par}
		\vfill
		{\Large\scshape Project \par}
		\vfill
		\par
		\vfill
		{\large \today\par}
	\end{titlepage}
	\newpage
\section{Introduction}
Nothing for now
%	\subsection{Motivation}

%
%  \section{Overview}
%  \begin{table}[H]
%    \centering
%    \begin{tabular}{l|lll}
%      \textbf{Langauge}  & Name & Mode & Implementation  \\ \hline
%      Julia                & JuliaDiff\footnote{https://www.juliadiff.org/} &  F/R  & OO \\
%      Swift (TensorFlow?)  & &     & \\
%      Tensorflow           & Tensorflow   & R  & OO \\
%      STAN                 &   &  & OO \\
%    \end{tabular}
%    \caption{Overview of languages}
%    \label{tab:overview}
%  \end{table}
%
%\subsection{Julia}
%In forwardDiff it uses extended dual numbers for higher order dimensions 
%
%
%
%\subsection{Tensorflow}
%
%Reverse-mode using gradient tape


\section{Sparsity}
This section is based on Julia's approach to sparse AD computation. 
Link: \url{https://github.com/JuliaDiffEq/SparseDiffTools.jl}. 


\subsection{Example: Data-independent sparsity}

Below is a function which performs a 1D-convolution on the input array $x$ of dimension $n$. 
The Jacobian of such a function is the tri-diagonal matrix shown in Figure \ref{fig:tridiag}. 
\begin{minted}{C}
int* f (int* x, int n) {
  retval = malloc(sizeof(int) * n);
  for (int i = 1; i < n - 1; i++) {
    retval[i] = x[i-1] + x[i] + x[i+1];
  }
  retval[0] = x[0] + x[1];
  retval[n-1] = x[n-1] + x[n-2];
  return retval;
}
\end{minted}
\begin{figure}[H]
	$$ J_{f} = \left(\begin{matrix}
	\frac{\partial f_1}{\partial x_1} & \frac{\partial f_1}{\partial x_2} & 0 & 0 & 0 \\
	\frac{\partial f_2}{\partial x_1}& \frac{\partial f_2}{\partial x_2} & \frac{\partial f_2}{\partial x_3} & 0 & 0\\
	0 & \frac{\partial f_3}{\partial x_2} & \frac{\partial f_3}{\partial x_3} & \frac{\partial f_3}{\partial x_4} & 0\\
	0 & 0 & \frac{\partial f_4}{\partial x_3} & \frac{\partial f_4}{\partial x_4} & \frac{\partial f_4}{\partial x_5} \\
	0 & 0 & 0 & \frac{\partial f_5}{\partial x_4} & \frac{\partial f_5}{\partial x_5}
	\end{matrix}\right) $$
	\caption{Jacobian from \texttt{f} for $n=5$}
	\label{fig:tridiag}
\end{figure}
The sparsity of the Jacobian in this case is independent of the data, e.g. $x_1$ will always be independent of $f_5$ 
or more generally the off tri-diagonal elements will always be zero. 
Of course the case where the input value is zero, e.g. $x_1 = 0$  the entry $\frac{\partial f_1}{\partial x_1}$ will also be zero, 
which would then be \emph{data} dependent.

\subsection{Example: Bundle Adjustment - Data-dependent}
For a data-dependent example we use  Bundle Adjustment \footnote{Link to Wiki} as example and 
show that the layout of the Jacobian depends on how we represent our data as shown in Figure \ref{fig:jacobian-matrix-for-a-bundle-adjustment-problem-con-sisting-of-3-cameras-and-4-points}.
Note that for showing the natural sparsity of the Jacobian we have to represent the input data as one long vector
i.e. our input data has the form $\lbrack \texttt{cam1} \ \texttt{cam2} \ \texttt{cam3}\ X_1\ X_2\ X_3\ X_4 \rbrack $, 
where only one pair of \texttt{camX} and $X_i$ are non-zero for each data-row. 
The Jacobian sparsity pattern hence depends on the input, for example if we swap the columns \texttt{cam1} and \texttt{cam2}, keeping all else the same,
the sparsity pattern of the Jacobian also changes. 
\begin{figure}[H]
	\centering
	%\includegraphics[width=0.7\linewidth]{../../../Desktop/Jacobian-matrix-for-a-bundle-adjustment-problem-con-sisting-of-3-cameras-and-4-points}
	\caption{Data dependent sparse matrix stemming from Bundle Adjustment (Stolen from  \url{https://www.researchgate.net/figure/Jacobian-matrix-for-a-bundle-adjustment-problem-con-sisting-of-3-cameras-and-4-points_fig1_284154527})}
	\label{fig:jacobian-matrix-for-a-bundle-adjustment-problem-con-sisting-of-3-cameras-and-4-points}
\end{figure}


\subsection{Sparsity w.r.t. performance}
For Forward-mode AD we can reduce the computation of both types of sparsity with a \texttt{colour} vector, 
which  denotes which input parameters are independent (or dependent) of each other. 
This corresponds to solving a graph-colouring problem.
For example for the function \texttt{f} above we can use a color vector 
with three colours as each output value depends on at most three input values. 
A colour vector for $n=5$ could then be $\lbrack c_0, c_1, c_2, c_1, c_0\rbrack$ where input values with the 
same colour can be computed during the same forward-sweep, so we reduce the number of 
forward calls to the number of colours used. Worth noting is that \texttt{f}'s colour vector 
 \emph{always}  contains 3 colours; independent of $n$ so the number of forward calls 
 is effectively reduced by  $n-3 + 1$ in this case. (Plus one since we need to compute the colour vector)\newline 
 For finding the colour-vector we can perform a forward sweep of the function 
 with some random input and use some underlying data-structure to solve the graph-colouring problem. 
 Alternatively can one let the user provide the color vector. 
 
 \subsubsection{Bundle Adjustment revisited}
 For Bundle Adjustment we can also use the same approach by seeding the forward sweep 
 with the camera and point data. Using the data it's possible to figure out the dependencies and generate a 
 coloring, which in this case would be a matrix, one for each data-row. 
 In stochastic gradient descent methods, like used Bundle Adjustment, this color vector can provide a speed-up on subsequent 
 computations of the Jacobian. 
 However if the data is only used once, this approach requires that we compute the coloring again and again
 and hence would not be very efficient. 



\subsection{Sparsity w.r.t. Memory}
Julia and STAN uses a sparse data structures but to my knowledge not feasible in Futhark?
Need more time to look into this. 


\newpage
\nocite{*}
\printbibliography


\end{document}