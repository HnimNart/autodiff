\documentclass[12pt, a4paper]{article}
\usepackage [T1]{fontenc}
\usepackage[margin=2.5cm]{geometry}
\usepackage{graphicx}
\usepackage[utf8]{inputenc}
\usepackage[english]{babel}
\usepackage{bm}
\usepackage{float}
\usepackage{pifont}

\usepackage[toc,page]{appendix}
\usepackage[nottoc]{tocbibind}
\setcounter{tocdepth}{1}


\usepackage{algorithmic}

\usepackage[normalem]{ulem}
\usepackage{multirow}
\usepackage{tcolorbox}
\usepackage{adjustbox}
\usepackage{array}
\usepackage{tabularx}
\usepackage{longtable}
\usepackage{tabulary}
\usepackage{color}
\usepackage{colortbl}
\usepackage{gensymb}
\usepackage{pdfpages}
\usepackage{caption}
\usepackage{subcaption}
\usepackage{diagbox}
\usepackage{hyperref}
\usepackage[section]{placeins}
\usepackage{bbm}
\usepackage[linesnumbered,algoruled]{algorithm2e}

%% Biblography
\usepackage[style=numeric-comp,
            hyperref=auto,
            backend=biber,
            backend=bibtex]{biblatex}


% Code format
\usepackage{listings}
\usepackage{minted}
\setminted{
	frame=lines,
	framesep=2mm,
	baselinestretch=1.2,
	linenos,
	fontsize=\footnotesize,
	breaklines=true
}

% Math ops
\usepackage{mathtools}
\usepackage{amsmath, amssymb}
\DeclareMathOperator*{\argmax}{arg\,max}
\DeclareMathOperator*{\argmin}{arg\,min}
\DeclareMathOperator*{\EX}{\mathbb{E}}% expected value

\DeclarePairedDelimiter{\ceil}{\lceil}{\rceil}
\DeclarePairedDelimiter{\floor}{\lfloor}{\rfloor}

\newcommand{\dunderline}[1]{\underline{\underline{#1}}}
\newcommand{\euler}[1]{\ensuremath{\ \mathrm{e}^{#1}}}
\newcommand{\ma}[1]{\textbf{#1}}
\newcommand{\ve}[1]{\textbf{#1}}
\newcommand{\ap}[1]{\mathrm{\overline{#1}}}
\newcommand{\pp}[1]{\left( {#1} \right)}
\newcommand{\abs}[1]{\left|   {#1} \right|  }
\newcommand{\forn}[2]{{#1} \rightarrow {#2} \quad \text{for} \quad n \rightarrow \infty }
\newcommand{\forN}[2]{{#1} \rightarrow {#2} \quad \text{for} \quad N \rightarrow \infty }
\newcommand{\folg}[3]{ \{#1_{#3}\}_{#3 = #2}^{\infty} }
\newcommand{\folgc}[3]{ \{#1_{#3}\}_{#3 = #2}^{c} }
\newcommand{\ip}[2]{\langle{#1} , {#2}\rangle}

\newcommand{\RR}{ \mathbb{R} }
\newcommand{\NN}{ \mathbb{N} }
\newcommand{\CC}{ \mathbb{C} }
\newcommand{\for}{  \quad \text{for} \quad  }
\newcommand{\as}{  \quad \text{as} \quad  }

\usepackage{amsthm}
\usepackage{xcolor}


\theoremstyle{definition}
\newtheorem{definition}{Definition}[section]
\newtheorem{problem}{Problem}[section]
\newtheorem{theorem}{Theorem}[section]
\newtheorem{corollary}{Corollary}[theorem]
\newtheorem{lemma}[theorem]{Lemma}
\usepackage{hyperref}
\addbibresource{main.bib}
\author{Minh}
\begin{document}
	\begin{titlepage}
		\centering
		\includegraphics[width=\textwidth]{KuLogo.png}
		\par\vspace{1cm}
		\vspace{1cm}

		{\scshape\Large Automatic Differentiation \\
			\par}

		\vspace{1.5cm}
		{\Large\itshape  Minh Duc Tran }\\
		\vspace{0.5cm}
		{\scshape\large cwz688 \\ \par}
		\vfill

		{\Large\scshape Master project \par}

		\vfill
		\par
		\vfill
		{\large \today\par}
	\end{titlepage}
	\newpage
	\tableofcontents
	\newpage
	\section{Introduction}
	Many important algorithms require taking derivatives to be solved. Such algorithms
	include Bundle Adjustment\footnote{\url{https://en.wikipedia.org/wiki/Bundle_adjustment}} or training neural networks using backpropagation. 
	These algorithms often involves finding an optimum of a function. 
	Finding the derivative for functions can always be done by hand. 
	While this method is accurate and often the fastest method when executed, 
    it scales poorly as finding the derivatives by hand is time consuming and can be error prone.  We
	hence seek to automate the process of finding these derivatives. \newline

	The simplest method to implement is \textit{numerical differentiation}
	but is slow in execution as it requires two evaluations per scalar derivative.
	It also suffers from numerical imprecision. An exact method is
	\textit{symbolic differentiation}, which are often used in  ''mathematical''
	languages
	like Matlab, Mathematica or by your old TI-89.  This generates exact symbolic
	derivatives, but
	is memory intensive and slow to compute. It can also not handle
	complex logic such as unbounded loops and hence is not appropriate for general
	computer programs.
	For solving most of these issues we turn to \textit{automatic differentiation},
	which can
	compute exact derivatives for an arbitrary computer program. The caveat is
	however
	that the implementation must be carefully thought out. We will in this report
	see how one can implement automatic differentiation in a way that reduces
	the \emph{computational} cost when performing forward-mode automatic differentiation. 
	While memory usage also can be a problem for such implementation it is not the main 
	concern of this report. \newline
	
	
	 
	Before we dive into the optimizations, the next section first provides a general introduction to 
	automatic differentiation. After that we introduce the optimizations which is 
	based on matrix partitioning using graph colouring techniques. A small implementation
	in Futhark is then shown for a simple example to see how simple one can use the shown method
	as an addition to an already implemented AD library. 
	


	\section{Automatic differentiation}
	Automatic differentiation (AD) relies on the fact every computer program,
	regardless of how complex it is,
	is a composition of arithmetic operations, like addition, multiplication etc.
	with well-defined derivatives.
	Along with the chain rule one can compute the derivative for any differentiable
	function written as a computer program.
	For example consider the composition of $n$ scalar functions:
	$f(x) = f^n \circ f^{n-1} \circ \cdots \circ f^1 \circ x$. The chain rule
	then simply states that the derivate of
	$f$ w.r.t. $x$ is
	\begin{equation}
	\frac{\partial f}{\partial x} = \frac{f^L}{f^{L-1}} \frac{f^{L-1}}{f^{L-2}}
	\cdots \frac{f^1}{x}
	\label{eq:f}
	\end{equation}
	By formulating our function $f(x)$ as a program we are able to use AD to compute
	$\frac{\partial f}{\partial x}$ automatically.
	AD implementations usually comes in two distinct modes, forward- and reverse-mode.
	The main difference between these two is in which direction the terms of each
	partial derivatives of the chain rule
	are computed. For example in Eq. \ref{eq:f} forward-mode starts from the right
	most term, i.e. it first computes $\frac{f^1}{x}$
	and traverses to the left. On the other hand does reverse-mode start from
	$\frac{f^L}{f^{L-1}}$ and accumulate to the right. \newline
	The formulating above is a bit simplified for most problems and in the
	general case have a multi-dimensional function $f:\mathbb{R}^n \to \mathbb{R}^m$  which
	would then have the more general $m\times n$ derivative called the Jacobian matrix $J_f$ defined by:
	\begin{figure}[H]
		$$ J_{f} = \left(\begin{matrix}
		\frac{\partial f_1}{\partial x_1} & \cdots & \frac{\partial f_1}{\partial x_n}
		\\
		\vdots & \ddots & \vdots \\
		\frac{\partial f_m}{\partial x_1} & \cdots  &  \frac{\partial f_m}{\partial
			x_n}\\
		\end{matrix}\right) $$
		\caption{$m\times n$ Jacobian matrix of layout $f : \mathbb{R}^n \to
			\mathbb{R}^m$}
	\end{figure}
    Most algorithms require the computation of the Jacobian. We first look at forward-mode.
    Given an input vector $x \in \mathbb{R}^n$  forward-mode evaluates $f$ once and
	computes a \emph{column} of $J_f$. Hence to compute the entire Jacobian we need 
	to perform $n$ evaluations of $f$. This means that forward-mode scales with $\mathcal{O}(n)$
	and hence for  large functions with many input variables, forward-mode AD can quickly become a very 
	expensive method for computing the Jacobian. 
	Techniques for reducing the number of evaluations can be done by performing an
	analysis on the columns of the Jacobian and partitioning them into as few structurally orthogonal groups as
	possible.
	This technique will be described in more detail in Section \ref{sec:partitioning}.\newline 
	Reverse-mode works in slightly different way. 
	Given an output space vector $y
	\in \mathbb{R}^m$, a reverse-sweep will compute a
	\emph{row} of $J_f$. Hence do we need to perform 1 evaluation of $f$ and $m$
	reverse-mode sweeps, one for each scalar value in $y$, for computing the Jacobian. 
	Reverse-mode therefore scales linearly as $\mathcal{O}(m)$.  
	We can now see the computational difference between the two modes. When $n \ll m$ forward-mode is
	more efficient and vice versa for $n \gg m$.
	Applications such as training a neural networks usually have large input
	dimensions $n$ compared to output dimension $m$ and hence are
	libraries like Tensorflow\footnote{\url{https://www.tensorflow.org/}}
	reverse-mode based. One downside to reverse-mode is that
	it requires access to intermediate variables computed during the evaluation and
	hence require more memory. As Tensorflow is a DSL for Machine Learning applications
	it can aggregate the intermediate values into fewer objects which represents a
	higher level of abstraction. For example are layers in neural network composed of
	many multiplications, but instead of storing the intermediate value from each multiplication
	the result is aggregated into few intermediate value. Hence will specialized reverse-mode AD
	be a more efficient than a general purpose reverse-mode framework, as it can leverage
	domain-specific structures. For general purpose reverse-mode AD memory requirements
	tend to explode as each intermediate variables needs to store it's numerical value until the derivative
	has been evaluated during the reverse-sweep. 
	For e.g. loops, this can quickly explode if not carefully implemented.  
	Frameworks for performing efficient reverse-mode AD is an active area of research 
	and several techniques such as retaping and checkpoints\cite{Margossian2018ARO} 
	can be applied to reduce the peak memory usage. \newline 	
	For forward-mode AD the memory requirement is simply twice the memory
	requirement of the initial function\footnote{It will be clearer in the next section why this is}. 
	Because of the simpler nature of optimizing forward-mode AD implementations with much simpler 
	optimization techniques, we will in the remaining report focus on forward-mode. 

	\subsection{Forward-mode}
	This section describes how forward-mode can be implement and provide the optimization
	used for reducing the computational cost. 
	The implementation of forward-mode can be done with two strategies:
	\subsubsection*{Source code transformation}
	One of the oldest methods for AD is source code transformation. Here an AD tool,
	(or preprocessor), rewrites our function that we wish to
	compute the derivative of. A preprocessor then applies the differentiation rules
	as prescribed by the elementary operators and chain rules,
	and returns the augmented source code function which now includes statements for
	computing the derivatives. The resulting source
	code can be compiled and executed as usual. There are few examples of such
	implementations.
	For example has Tangent\cite{DBLP:journals/corr/abs-1711-02712} implemented AD
	support for a subset of the Python language,
	while Tapenade has AD tools for transforming C and Fortran functions. The major
	downsides of source code transformation is that
	they can only use information available at compile time and developing the tools
	is non-trivial and requires a lot of effort.
	Therefore do we often see that AD are implemented using operator overloading.

	\subsubsection*{Operator overloading}
	Operator overloading (OO) is far more common and simpler to implement.
	Forward-mode OO is usually implemented using \emph{dual numbers}, which extends
	the real numbers
	of the programming language by a differential component. The real number
	operators are then lifted to
	operate on the dual numbers and provide a powerful abstraction, where the
	details of derivatives can
	be hidden away from the programmer. Furthermore can implementations usually be
	an embedded DSL in form of a library
	within a more general host language. These DSL can then leverage the constructs
	and features of the host language and can be imported
	as any other library. This ensures a flexible and portable advantage of having
	such an embedded DSL. Examples of such DSL libraries are
	\texttt{Julia ForwardDiff}\cite{RevelsLubinPapamarkou2016} and \texttt{Stan
		Math}\cite{DBLP:journals/corr/CarpenterHBLLB15}.
	The next section provides an example of how forward-mode AD with operator
	overloading can be applied to a simple
	function.




	\subsubsection*{Example}
	Consider the  function $f:\mathbb{R}^2 \to \mathbb{R}$ defined by $f(x_1, x_2) =
	x_1 \cdot x_2 + \sin x_1$.
	We show how one can numerical compute the partial derivative
	$\frac{\partial}{\partial x_1} \left( x_1 \cdot x_2 + \sin x_1 \right)$
	of such a function by using forward-mode AD with operator overloading. With
	operator overloading each
	variable $x$ is augmented with it's \emph{numerical} derivate value, which we
	denote by a dot, i.e. for a variable $x_1$
	we denote it's derivate value as $\dot{x}_1$. The operations that can be
	performed on such a variable
	are now a function pair of the operation itself and it's derivative expression
	w.r.t. it's inputs. For example for
	scalar multiplication on $x_1$ and $x_2$ the  derivative expression is $x_1
	\cdot \dot{x}_2 + x_2 \cdot \dot{x}_1$ as we want to be able to compute the
	derivative of either $x_1$ or $x_2$. Hence we have that a scalar value is stored
	as $(x, \dot{x})$ and a multiplication
	operation on such a type has the operator pair $(x_1 \cdot x_2; x_1 \cdot
	\dot{x}_2 + x_2 \cdot \dot{x}_1) $.
	As the function is evaluated with forward-mode AD,  also called a forward sweep,
	the derivatives are then automatically evaluated along and combined through the
	chain rule.
	The independent variable which one wants to differentiate with respect to is
	decided by setting a \emph{seed} for that variable.
	For example as we want to differentiate w.r.t. $x_1$ we set $\dot{w}_1 =
	\frac{\partial x_1}{\partial x_1} = 1$ and $\dot{w}_2 = \frac{\partial
		x_2}{\partial x_2} = 0$, where we define $\dot{w}_i$ for sub-expression $i$.
	With these seeds set; Table \ref{tab:example}  shows a symbolic forward sweep of
	$f$ for computing $\frac{\partial}{\partial x_1} \left( x_1 \cdot x_2 + \sin x_1
	\right)$.
	%Some of the derivate identities used in this example are $\frac{\partial
	%x}{\partial x} = 1$ and $\frac{\partial \sin x}{\partial x} = \cos x$.
	\begin{table}[H]
		\centering
		\begin{tabular}{l|l}
			Operations to evaluate $f$ & Operations to evaluate the derivative          \\
			\hline \hline
			$w_1 = x_1$               & $\dot{w}_1 = 1$ (seed)
			\\
			$w_2 = x_2$               & $\dot{w}_2 = 0$ (seed)
			\\
			$w_3 = w_1 \cdot w_2$     & $\dot{w_3} = w_2 \cdot \dot{w}_1 + w_1 \cdot
			\dot{w}_2$ \\
			$w_4 = \sin w_1$          & $\dot{w}_4 = \cos w_1  \cdot  \dot{w}_1$
			\\
			$w_5 = w_3 + w_4$         & $\dot{w}_5 = \dot{w}_3 + \dot{w}_4$

		\end{tabular}
		\caption{Example of a symbolic forward pass for forward-mode AD for computing
			$\frac{\partial}{\partial x_1} \left( x_1 \cdot x_2 + \sin x_1 \right)$}
		\label{tab:example}
	\end{table}
	By setting the input values, $x_1 = 1$ and $x_2 = 2$ we can numerically evaluate
	our derivative as:
	\begin{equation}
	\frac{\partial f}{\partial x_1} = w_2 \cdot \dot{w}_1 + w_1 \cdot \dot{w}_2 +
	\cos w_1  \cdot  \dot{w}_1 =  2 \cdot 1 + 1 \cdot 0 + \cos 1 \cdot 1 = 2.5403
	\label{eq:seeds}
	\end{equation}
	Which is the result we would also have obtained through differentiation by hand.

	If we wanted to compute the Jacobian of $f$ we also need to compute
	$\frac{\partial f}{\partial x_2}$
	and we would need to change our seeds to $\dot{w}_1 = 0$ and $\dot{w}_2 = 1$
	and perform another forward sweep.
	The reader may wonder why we can't set both seeds to 1 and settle for single
	forward sweep.
	Consider the operation $x_1 \cdot x_2$ with the two input values, $x_1$ and
	$x_2$ with the derivative expression
	$x_1 \cdot \dot{x}_2 + x_2 \cdot \dot{x}_1$. If we then set both seeds to 1,
	both terms will then contribute to
	the partial derivative, i.e. there is interleaving, which would result in
	neither derivatives being correct. For example in Eq. \ref{eq:seeds}
	will the middle term now contribute to the resulting derivative for
	$\frac{\partial f}{\partial x_1}$. Hence do we need a separate forward sweep for
	each
	independent variable that we wish to compute the derivative of.
	\subsubsection{Exploiting sparsity of the Jacobian}
	One common problem for many applications of AD is that many elements of the
	Jacobian
	are zeros and as such said to be sparse. Such sparsity in the Jacobian can
	be exploited for more efficient computations by using fewer forward passes. The
	problem
	of reducing the number of forward passes can be formulated as a matrix
	partitioning problem.
	As an example consider the sparse Jacobian for a function $f: \mathbb{R}^n \to
	\mathbb{R}^m$ with $m=n=5$ in Figure \ref{fig:jacobian}. By not exploiting the
	sparsity of the Jacobian, forward-mode AD requires $n=5$ forward passes to
	compute the Jacobian.
	\begin{figure}[H]
		$$ \left(\begin{matrix}
		j_{11} & j_{12} & 0 & 0 & j_{15} \\
		0 & 0 & j_{23} & 0 & 0 \\
		0 & j_{32} & j_{33} & j_{34} & 0 \\
		j_{41} & 0 & 0 & 0 & 0 \\
		0 & 0 & 0 & j_{54} & j_{55}
		\end{matrix}\right) $$
		\caption{Jacobian from \texttt{f} for $m=n=5$}
		\label{fig:jacobian}
	\end{figure}
	However it's possible in this case to exploit the sparsity to reduce the number
	of forward passes down to three. Consider a subset of the columns of the
	Jacobian such that no two columns have a non-zero in a common row. Such a subset
	of columns are said to be \emph{structurally orthogonal}. Formulated
	differently, is every pair of columns in such a subset, pairwise orthogonal to
	each other and is independent of each other in the numerical values of the
	non-zeros. In the example above are columns 1 and 3 structurally orthogonal, as
	well as columns 1 and 4; and 3 and 5. By finding the minimum number of
	structurally orthogonal subsets we are able to reduce the number of forward
	passes.
	The problem can be formulated as:
	\begin{problem}
		Given the sparsity structure of an $m \times n$ matrix A, find a structurally
		orthogonal partition of its columns that has the \emph{fewest} groups.
		\label{prob:p1}
	\end{problem}
	For example if we partition the Jacobian above into 3 subsets $\{\{1,4\}, \{2\},
	\{3,5\}\}$ we are able to compute the compressed Jacobian with just 3 passes.
	Without loss of information our compressed Jacobian will look like:
	\begin{figure}[H]
		$$ \left(\begin{matrix}
		{\color{blue} j_{11}} & {\color{red} j_{12}} & {\color{green} 0 }        &
		{\color{blue} 0 }       & {\color{green} j_{15}} \\
		{\color{blue} 0}        & {\color{red} 0 }       & {\color{green} j_{23}}  &
		{\color{blue} 0 }      & {\color{green} 0} \\
		{\color{blue} 0}        & {\color{red}j_{32}} &  {\color{green} j_{33}}  &
		{\color{blue} j_{34}} & {\color{green} 0 } \\
		{\color{blue} j_{41}} & {\color{red} 0 }       & {\color{green} 0 }         &
		{\color{blue} 0 }        & {\color{green} 0} \\
		{\color{blue} 0 }       & {\color{red} 0 }       & {\color{green} 0 }         &
		{\color{blue} j_{54}} & {\color{green} j_{55} }
		\end{matrix}\right) \Rightarrow \left(\begin{matrix}
		{\color{blue} j_{11}} & {\color{red} j_{12}} & {\color{green} j_{15}} \\
		{\color{blue} 0} & {\color{red} 0} & {\color{green} j_{23}} \\
		{\color{blue} j_{34}} & {\color{red} j_{32}} & {\color{green} j_{33}}  \\
		{\color{blue} j_{41}} & {\color{red} 0} & {\color{green} 0}  \\
		{\color{blue} j_{51}} & {\color{red} 0} & {\color{green} j_{55}} &
		\end{matrix}\right) $$
		\caption{Compressed Jacobian from \texttt{f} for $m$ = $n$ = $5$ with
			partitioning $\{\{1,4\}, \{2\}, \{3,5\}\}$}
		\label{fig:Jacob_partition}
	\end{figure}
	In practice it means that we've reduced the size of the Jacobian from $5 \times
	5$ to a compressed Jacobian of size $3 \times 3$ and in this case it's clear
	that it's the minimum size that we can reduce to. Finding such an optimal
	partitioning can be done
	by modelling the problem as a graph colouring problem.  Unfortunately is
	the general graph-colouring problem known to be NP-hard, so algorithms presented
	here are based on effective heuristics. Next section presents simple algorithms
	for matrix partitioning of Jacobian. For a more thorough presentation it is
	refereed to \cite{Jacobian}, which also presents algorithms for
	partitioning Hessians and much more. The next section shows how we can formulate
	Problem \ref{prob:p1} as a
	Graph Colouring problem and presents algorithms for unidirectional\footnote{Here
		unidirectional means only partitioning over columns, unlike bidirectional which
		is partitioning over rows and column} matrix partitioning of Jacobian matrices.


	\subsection{Graph Colouring for matrix partitioning}
	\label{sec:partitioning}
	In the following I assume that the reader is familiar with basic graph theory
	terminology.
	This section's results are based on Chapter 1-3 in \cite{Jacobian} and here
	we  only present the most essential definitions and results used in the
	algorithm presented
	below for matrix partitioning with graph colouring. The proofs of the presented
	theorems and lemmas
	can also be found in \cite{Jacobian} and are here omitted for brevity.
	We start with a graph colouring definition.
	\subsubsection*{Distance-k Graph Colouring}
	\begin{definition}
		A \textit{distance-k} vertex colouring of a graph $G = (V,E)$ is a mapping $\phi
		: V \to \{1,2,..., p\}$ such that $\phi(u) \not= \phi(v)$ whenever vertices $u$
		and $v$ are distance-$k$ neighbours. The least possible number of colours
		required to colour $G$ is called the chromatic number of $G$ and is denoted
		$\chi_k(G)$. Furthermore is a distance-$k$ colouring of $G = (V,E)$ called
		\emph{partial} if it only covers a proper subset $W \subset V$ of the vertices.
		We then have that $\phi: W \to \{1,2, \cdots, p\}$ is a mapping such that
		$\phi(u) \not= \phi(v)$ whenever vertices $u, v \in W$ are distance-$k$
		neighbours in $G$.
	\end{definition}

%	\subsubsection*{The power of a Graph}
%	\begin{definition}
%		The \emph{k}th power of a graph $G$ is the graph $G^k$ whose vertex set is the
%		same as that of $G$ and whose edge set consists of pairs of vertices $(u,v)$
%		whenever vertices $u$ and $v$ are distance-$k$ neighbours in G.
%		This leads us to the  lemma:
%	\end{definition}
%	\begin{lemma}
%		A mapping $\phi : V \to \{1,2,..., p\}$ is a distance-$k$ coloring of $G$ if and
%		only if it is a distance-1 colouring of $G^k$
%	\end{lemma}

	\subsubsection*{Representing Matrices Using Graphs}
	%Here's we will present different graph presentations of the sparsity structure
	%of matrices.
	For a matrix $A$ we'll denote the $i$th row as $r_i$ and the $j$th column
	as $c_j$. Finally is the $(i,j)$ entry of A denoted as $a_{ij}$. We now show how
	we can
	construct a bipartite graph representation from a matrix $A$. \newline
	\textbf{Bipartite Graphs}: Let $A$ be an $m \times n$ matrix with rows $r_1,
	r_2, \dots, r_m$ and columns $c_1, c_2, \dots, c_n$. The bipartite graph of $A$,
	denoted $G_b(A)$, is defined as $G_b(A) = (V_1, V_2, E)$, where $V_1 = \{r_1,
	r_2, ..., r_m\}$, $V_2 = \{c_1, c_2, ..., c_n\}$ and $(r_i, c_j) \in E$ whenever
	$a_{ij}$ is non-zero for $1 \leq i \leq m, 1 \leq j \leq n$.  An example of the
	bipartite representation is shown in Figure \ref{fig:bipartite}.
	\begin{figure}[H]
		\centering
		\begin{subfigure}{0.45\linewidth}
			$$ \left(\begin{matrix}
			{\color{blue} a_{11}} & {\color{red} a_{12}} & {\color{green} 0 }        &
			{\color{blue} 0 }       & {\color{green} a_{15}} \\
			{\color{blue} 0}        & {\color{red} 0 }       & {\color{green} a_{23}}  &
			{\color{blue} 0 }      & {\color{green} 0} \\
			{\color{blue} 0}        & {\color{red} a_{32}} &  {\color{green} a_{33}}  &
			{\color{blue} a_{34}} & {\color{green} 0 } \\
			{\color{blue} a_{41}} & {\color{red} 0 }       & {\color{green} 0 }         &
			{\color{blue} 0 }        & {\color{green} 0} \\
			{\color{blue} 0 }       & {\color{red} 0 }       & {\color{green} 0 }         &
			{\color{blue} a_{54}} & {\color{green} a_{55} }
			\end{matrix}\right) $$
		\end{subfigure}
		$\Rightarrow$ \hfill
		\begin{subfigure}{0.45\linewidth}
			\includegraphics[width=0.6\textwidth]{pics/bipartite.png}
		\end{subfigure}
		\caption{Example of a bipartite representation of the matrix partitioning
			from Figure \ref{fig:Jacob_partition}}
		\label{fig:bipartite}
	\end{figure}
	Note that the size of graph is proportional to size of an $m \times n$ matrix A,
	i.e. we have $\vert V_1 \vert  + \vert V_2 \vert = m + n$ vertices and $\vert E
	\vert = nnz(A)$, where $nnz(A)$ denotes the number of non-zero entries of $A$.
	Having set our preliminaries we are now able to express Problem
	\ref{prob:p1} as a Graph Colouring Problem where  we make us of the following theorem:
	\begin{theorem}
		Let $A$ be an $m \times n$ matrix and $G_b(A) = (V_1, V_2, E)$ be its bipartite
		graph representation. A mapping $\phi$ is a partial distance-2 coloring of
		$G_b(A)$ on $V_2$ if and only if $\phi$ induces a structurally orthogonal
		partition of the columns of A.
		\label{th:th1}
	\end{theorem}
	The reader may verify that the partitioning in Figure \ref{fig:bipartite} indeed
	satisfies the partial distance-2 colouring of
	it's bipartite representation. An important corollary of Theorem \ref{th:th1} is

	\begin{corollary}
		Let $\chi_2(G_b, V_2)$ denote the chromatic number for a partial distance-2
		colouring of $G_b(A) = (V_1, V_2, E)$  on $V_2$.
		Then $\chi_2(G_b, V_2)$ is the minimum number of forward passes required to
		compute the Jacobian and hence is a lower bound.
		\label{col:minimum}
	\end{corollary}
	Corollary \ref{col:minimum} simply states that if we can find the minimum number
	of colours required for a partial distance-2 colouring of $G_b(A) = (V_1, V_2,
	E)$  on $V_2$  then we've found the optimal partition for computing the
	Jacobian in the fewest number of forward passes. From Theorem \ref{th:th1} we
	are now able to formulate the equivalent graph colouring problem from Problem
	\ref{prob:p1}:
	\begin{problem}
		Given the bipartite graph $G_b(A) = (V_1, V_2, E)$ representing the sparsity
		structure of an $m \times n$ matrix $A$, find a partial distance-2 colouring of
		$G_b(A)$ on $V_2$ that uses the fewest colours.
	\end{problem}
	Having defined our problem as graph colouring problem we are 
	now able to see how we can implement this technique. We'll first need 
	an algorithm for computing the graph colouring. We use a simple greedy approximation
	algorithm for this. 

	\subsubsection{A greedy distance-2 Colouring Algorithm}
	An algorithm for finding a greedy distance is shown in Algorithm
	\ref{alg:greedy2}. The algorithm is fairly simple. It iterates through each of the vertices 
	in $V_2$ and gives it the minimum free colouring that is not among it's 
	neighbours. As this is an approximation algorithm it doesn't guarantee an optimal 
	solution, but only guarantees an upper bound on the number of colours. This algorithm never uses
	more than $d+1$ colours, where $d$ is the number of distance-2 neighbours in the graph. 
	\newline
	\begin{algorithm}[H]
		\SetAlgoLined
		\KwIn{$G_b = (V_1, V_2, E)$ }
		\KwResult{A partitioning of the co}
		Let $v_1, v_2, ..., v_{\vert V_2 \vert}$ be a given ordering of $V_2$\;
		Initialize \texttt{forbiddenColours} with some some value $a \not\in V_2$\;
		\For{$i \leftarrow 1$ \texttt{to} $\vert V_2 \vert$}
		{
			\ForEach{$w \in N_1(v_i)$}
			{
				\ForEach{\texttt{coloured vertex} $x \in N_1(w)$}
				{
					\texttt{forbiddenColours}$\lbrack \texttt{color} \lbrack x \rbrack \rbrack
					\leftarrow v_i$
				}
			}
			\texttt{colour}$\lbrack v_i \rbrack$ $\leftarrow \min \{ c > 0 :
			\texttt{forbiddenColours}\lbrack c \rbrack \not= v_i \}$
		}
		\caption{A greedy partial distance-2 colouring algorithm}
		\label{alg:greedy2}
	\end{algorithm}
 


	\section{Implementation in Futhark}
	To see how we can make use of the results from previous section
	in Futhark we implement the colouring algorithm and show how easily
	we can apply it to a Jacobian to reduce the number of forward-passes.
	Implementing graph colouring for matrix column partitioning is fairly simple. We
	also, without any modifications, use the already implemented forward-AD library,
	\url{https://github.com/diku-dk/futhark-ad}, which uses dual numbers and operator overloading. All the code can be found in
	Appendix ??.
	Consider the stencil function in Listing \ref{lst:tridiag} and note that every
	output value only depends on at most three different input values.
\begin{listing}[H]
\begin{minted}{haskell}
let f xs =
  let n = length xs
  in map  (\i -> if i == 0 then
                   f32_dual.(xs[0] + xs[1])
                 else if i == n - 1 then
                   let x1 = xs[n - 2]
                   let x2 = xs[n - 1]
                   in f32_dual.(x1 + x2)
                 else
                   let x1 = xs[i-1]
                   let x2 = xs[i]
                   let x3 = xs[i+1]
                   in f32_dual.(x1 + x2 + x3)) (iota (n))
\end{minted}
\caption{Stencil function with a tri-diagonal Jacobian}
\label{lst:tridiag}
\end{listing}
	The Jacobian of such a function when the input dimension is $n= 5$, is shown in
	Figure \ref{fig:tridiag}. Observe that
	the Jacobian is very sparse and only
	contains elements along the
	tri-diagonal and zeros elsewhere.
	\begin{figure}[H]
		$$ J_{f} = \left(\begin{matrix}
		\frac{\partial f_1}{\partial x_1} & \frac{\partial f_1}{\partial x_2} & 0 & 0 &
		0 \\
		\frac{\partial f_2}{\partial x_1}& \frac{\partial f_2}{\partial x_2} &
		\frac{\partial f_2}{\partial x_3} & 0 & 0\\
		0 & \frac{\partial f_3}{\partial x_2} & \frac{\partial f_3}{\partial x_3} &
		\frac{\partial f_3}{\partial x_4} & 0\\
		0 & 0 & \frac{\partial f_4}{\partial x_3} & \frac{\partial f_4}{\partial x_4} &
		\frac{\partial f_4}{\partial x_5} \\
		0 & 0 & 0 & \frac{\partial f_5}{\partial x_4} & \frac{\partial f_5}{\partial
			x_5}
		\end{matrix}\right) $$
		\caption{Jacobian from \texttt{f} in Listing \ref{lst:tridiag} for $n=5$}
		\label{fig:tridiag}
	\end{figure}
	For a simple implementation of forward-mode AD we'll normally use
	$n$ forward passes to compute the Jacobian, as each forward pass computes a
	single column of the Jacobian. 	 Let $x = \lbrack x_0, x_2, x_3, x_4, x_4 \rbrack$ denote our input vector then
	we will need to set a seed for each of the scalar values. With dual numbers our input for computing the Jacobian is:
		\begin{figure}[H]
		$$ \left(\begin{matrix}
(x_0, 1) & (x_1, 0) & (x_2, 0) & (x_3, 0) & (x_4, 0) \\
(x_0, 0) & (x_1, 1) & (x_2, 0) & (x_3, 0) & (x_4, 0) \\
(x_0, 0) & (x_1, 0) & (x_2, 1) & (x_3, 0) & (x_4, 0) \\
(x_0, 0) & (x_1, 0) & (x_2, 0) & (x_3, 1) & (x_4, 0) \\
(x_0, 0) & (x_1, 0) & (x_2, 0) & (x_3, 0) & (x_4, 1)
		\end{matrix}\right) $$
		\caption{Seeded input matrix for computing the Jacobian of Function in Listing \ref{lst:tridiag}}
	\end{figure}
    Here the first input row will compute the column $\frac{\partial f}{\partial x_0}$ of the Jacobian,
    the second  will compute $\frac{\partial f}{\partial x_1}$ and so on.
	While we are able to \texttt{map} over each row in Futhark and compute everything in parallel on a GPU
	most of the computations are wasted.
	By applying the colouring techniques from previous section, we able to only use
	$3$ forward passes
	of the function \texttt{f} \emph{regardless} of the input dimension $n$.
	Recall that we able to transform any $m\times n$ matrix $A$ into a bipartite
	graph
	representation by letting the row be set of vertices of $V_1$ and the columns be
	$V_2$.
	We use an adjacency matrix to represent our bipartite graph as such:
	\begin{figure}[H]
		$$ G_b(J_f) = \left(\begin{matrix}
		1 & 1 & 0 & 0 & 0 \\
		1 & 1 & 1 & 0 & 0\\
		0 & 1 & 1 & 1 & 0\\
		0 & 0 & 1 & 1 & 1 \\
		0 & 0 & 0 & 1 & 1
		\end{matrix}\right) $$
		\caption{Adjacency matrix, $G_b(J_f)$ derived from \texttt{$J_f$} in Figure
			\ref{fig:tridiag} with rows representing vertices of $V_1$ and columns
			representing vertices of $V_2$}
	\end{figure}
	The implementation of the greedy distance-2 colouring in Futhark is shown in
	Listing \ref{lst:greedy2}. Note that the implementation is sequential in the
	outer-most loop, but is parallel inside the loop-body.
	A  discussion of parallel colouring algorithms are beyond the scope of this
	project, but should be part of the future work of this project.
\begin{listing}[H]
\begin{minted}{haskell}
let greedy_distance_2_coloring [m][n] (max_num_colors:i32) (G:[m][n]i32) =
  let coloring = replicate n 0
  let colored = replicate (n*m) (false)
  let forbiddenColors = replicate (min max_colors n) (n+1)
  let (coloring, _, _) =
    loop (coloring, forbiddenColors, colored) for j < n do
      let N_v = filter (>=0) <| tabulate m (\i -> if G[i,j] == 1 then i else (-1))
      let colored_index_N_w = flatten <| map (\j' -> tabulate m (\i -> if unsafe colored[j' * m + i] then i else (-1))) N_v
      let colored_index_N_w = filter (>= 0) colored_index_N_w
      let color_idx = map (\i -> coloring[i]) colored_index_N_w
      let forbiddenColors' = scatter forbiddenColors color_idx (replicate (length color_idx) j)
      let idx = map (\i -> i * m + j) N_v
      let coloring[j] = argmin j forbiddenColors'
      let colored' = scatter colored idx (replicate (length idx) true)
      in (coloring, forbiddenColors', colored')
  in coloring
\end{minted}
\caption{Implementation of Algorithm \ref{alg:greedy2} in Futhark}
\label{lst:greedy2}
\end{listing}
	Let $C = \lbrack c_0, c_1, c_2, c_0, c_1 \rbrack$ denote the
	colouring of the columns obtained from the algorithm above.
	Since each output value only depend on at most 
	3 input values, it's fairly easy to see that this is optimal. From Corollary \ref{col:minimum}
	we've then found the lower bound of forward passes for computing Jacobian.  
	We are now able to compute the  seed matrix as such.
\begin{listing}[H]
\begin{minted}{haskell}
let compute_seed_matrix [n] (coloring:[n]i32) :[][]f32 =
  let max_col = maximum coloring
  in tabulate max_col (\i -> map (\c -> if c == i then 1 else 0) coloring)
\end{minted}
\end{listing}
Our new seeded input matrix then becomes:
\begin{figure}[H]
	$$ \left(\begin{matrix}
	(x_0, 1) & (x_1, 0) & (x_2, 0) & (x_3, 1) & (x_4, 0) \\
	(x_0, 0) & (x_1, 1) & (x_2, 0) & (x_3, 0) & (x_4, 1) \\
	(x_0, 0) & (x_1, 0) & (x_2, 1) & (x_3, 0) & (x_4, 0)
	\end{matrix}\right) $$
	\caption{Seeded input matrix for computing the Jacobian of Function in Listing \ref{lst:tridiag}}
\end{figure}
Now the first input row will compute \emph{both} $\frac{\partial f}{\partial x_0}$
\emph{and} $\frac{\partial f}{\partial x_3}$ during the same forward-pass.
With the new input matrix we are now able to compute the compressed Jacobian
with only 3 forward passes. Restoring the Jacobian from the compressed one
can be done by using the colouring and graph to reverse the indices. This example
shows how we can reduce the computational cost of forward-mode AD without 
modifying the existing implementation. This provides a flexible add-on 
to the existing framework. 


\section{Discussion}
While the given example results is optimal partitioning, which results in 
only requiring 3 forward passes regardless of the input size
it's a special case. In the general case we might not be so lucky. 
\subsection*{Initial cost}
We still need to perform full pass, i.e. $n$ evaluations of $f$, to obtain the 
initial Jacobian for creating the graph. Hence will this only be 
worth it when the colouring optimization is used multiple times. 
Whether the it's worth to find a colouring is obviously
problem dependent, but as most optimization
algorithms compute and apply the Jacobian over and over 
this might be worth the initial cost in these cases. Algorithms that might benefit 
include Bundle Adjustment and Backpropagation which performs the same optimization 
over the same data set over and over again. 

\subsection*{Graph Colouring algorithms (in Futhark)} 
The presented graph colouring algorithm is an approximation algorithm.
While it performs reasonable in practice it does not guarantee an optimal
colouring. Question is does it perform well enough? A deeper dive into
better (parallel) graph colouring algorithms is needed, which provides better guarantees 
on the solution. Another concern is the efficiency of the algorithm. As
the implemented algorithm is sequential in the input dimension for a
sufficiently large $n$ the computational time might become too large. 
I will leave this for future work to investigate these two issues in more depth. 


\section{Conclusion}

%
%%\subsection*{Data dependent Sparsity}
%%Consider the stencil function from Listing \ref{lst:tridiag}.
%%	The sparsity of the Jacobian in this case is independent of the data, e.g. $x_1$
%%	will always be independent of $f_5$	or more generally the off 
%%	tri-diagonal elements will always be zero.\newline 
%%As we use the resulting Jacobian from an initial forward pass, 
%%and use zero entries as an indication of 
%%columns being structurally orthogonal we might have a zero-entry at $a_{ij}$ because
%%of the data that was given during the initial forward pass. If we compute the initial 
%%Jacobian with some zero data 
%	
%	
%%	However does optimization problems use data. Such data 
%%	can induce a sparsity-pattern. 
%%	For a data-dependent example we use  Bundle Adjustment\footnote{\url{https://en.wikipedia.org/wiki/Bundle_adjustment}}
%%	as an example and
%%	show that the layout of the Jacobian depends on how we represent our data as
%%	shown in Figure
%%	\ref{fig:jacobian-matrix-for-a-bundle-adjustment-problem-con-sisting-of-3-cameras-and-4-points}.
%%	Bundle Adjustment (BA) is a classic problem in Computer Vision, in which from a set of cameras, 
%%	where images have been taken and selecting a fixed set of 2D points in the images,  
%%	is to recreate the scene in 3D. This is done by adjusting the camera parameters and
%%	the location of the 3D points. Hence is BA an optimization problem
%%	which iteratively computes projection of the 3D points onto the image plane 
%%	and computes the reprojection error, i.e.
%%	the difference between the projected 3D point on the image plane with the actual. 
%%	The Jacobian is then applied to minimize the reprojection error. 
%%	But because not all points are visible from all cameras, there will be some 
%%	
%%		\begin{figure}[H]
%%		\centering
%%		
%%		%\includegraphics[width=0.7\linewidth]{../../../Desktop/Jacobian-matrix-for-a-bundle-adjustment-problem-con-sisting-of-3-cameras-and-4-points}
%%		\caption{Data dependent sparse matrix stemming from Bundle Adjustment (Stolen
%%			from
%%			\url{https://www.researchgate.net/figure/Jacobian-matrix-for-a-bundle-adjustment-problem-con-sisting-of-3-cameras-and-4-points_fig1_284154527})}
%%		
%%		\label{fig:jacobian-matrix-for-a-bundle-adjustment-problem-con-sisting-of-3-cameras-and-4-points}
%%	\end{figure}
%%	Note that for showing the natural sparsity of the Jacobian we have to represent
%%	the input data as one long vector
%%	i.e. our input data has the form $\lbrack \texttt{cam1} \ \texttt{cam2} \
%%	\texttt{cam3}\ X_1\ X_2\ X_3\ X_4 \rbrack $,
%%	where only one pair of \texttt{camX} and $X_i$ are non-zero for each data-row.
%%	The Jacobian sparsity pattern hence depends on the input, for example if we swap
%%	the columns \texttt{cam1} and \texttt{cam2}, keeping all else the same,
%%	the sparsity pattern of the Jacobian also changes.
%	
%%	\item The presented graph colouring algorithm is an approximation algorithm.
%%	While it performs reasonable in practice it does rarely yield an optimal
%%	colouring
%
%
%
%
%
%
%
%\section{Conclusion}
%
%
%
%
%	\newpage
%	\newpage
%
%	\appendix
%	\section{Legacy stuff}
%	\subsection{Example: Data-independent sparsity}
%
%	This section is based on Julia's approach to sparse AD computation.
%	Link: \url{https://github.com/JuliaDiffEq/SparseDiffTools.jl}.
%
%	Below is a function which performs a 1D-convolution on the input array $x$ of
%	dimension $n$.
%	The Jacobian of such a function is the tri-diagonal matrix shown in Figure
%	\ref{fig:tridiag}.
%	\begin{minted}{C}
%	int* f (int* x, int n) {
%	retval = malloc(sizeof(int) * n);
%	for (int i = 1; i < n - 1; i++) {
%	retval[i] = x[i-1] + x[i] + x[i+1];
%	}
%	retval[0] = x[0] + x[1];
%	retval[n-1] = x[n-1] + x[n-2];
%	return retval;
%	}
%	\end{minted}
%	\begin{figure}[H]
%		$$ J_{f} = \left(\begin{matrix}
%		\frac{\partial f_1}{\partial x_1} & \frac{\partial f_1}{\partial x_2} & 0 & 0 &
%		0 \\
%		\frac{\partial f_2}{\partial x_1}& \frac{\partial f_2}{\partial x_2} &
%		\frac{\partial f_2}{\partial x_3} & 0 & 0\\
%		0 & \frac{\partial f_3}{\partial x_2} & \frac{\partial f_3}{\partial x_3} &
%		\frac{\partial f_3}{\partial x_4} & 0\\
%		0 & 0 & \frac{\partial f_4}{\partial x_3} & \frac{\partial f_4}{\partial x_4} &
%		\frac{\partial f_4}{\partial x_5} \\
%		0 & 0 & 0 & \frac{\partial f_5}{\partial x_4} & \frac{\partial f_5}{\partial
%			x_5}
%		\end{matrix}\right) $$
%		\caption{Jacobian from \texttt{f} for $n=5$}
%
%	\end{figure}
%	The sparsity of the Jacobian in this case is independent of the data, e.g. $x_1$
%	will always be independent of $f_5$
%	or more generally the off tri-diagonal elements will always be zero.
%	Of course the case where the input value is zero, e.g. $x_1 = 0$  the entry
%	$\frac{\partial f_1}{\partial x_1}$ will also be zero, which would then be
%	\emph{data} dependent.
%
%	\subsection{Example: Bundle Adjustment - Data-dependent}
%	For a data-dependent example we use  Bundle Adjustment \footnote{Link to Wiki}
%	as example and
%	show that the layout of the Jacobian depends on how we represent our data as
%	shown in Figure
%	\ref{fig:jacobian-matrix-for-a-bundle-adjustment-problem-con-sisting-of-3-cameras-and-4-points}.
%	Note that for showing the natural sparsity of the Jacobian we have to represent
%	the input data as one long vector
%	i.e. our input data has the form $\lbrack \texttt{cam1} \ \texttt{cam2} \
%	\texttt{cam3}\ X_1\ X_2\ X_3\ X_4 \rbrack $,
%	where only one pair of \texttt{camX} and $X_i$ are non-zero for each data-row.
%	The Jacobian sparsity pattern hence depends on the input, for example if we swap
%	the columns \texttt{cam1} and \texttt{cam2}, keeping all else the same,
%	the sparsity pattern of the Jacobian also changes.
%
%
%
%	\subsection{Sparsity w.r.t. performance}
%	For Forward-mode AD we can reduce the computation of both types of sparsity with
%	a \texttt{colour} vector,
%	which  denotes which input parameters are independent (or dependent) of each
%	other.
%	This corresponds to solving a graph-colouring problem.
%	For example for the function \texttt{f} above we can use a color vector
%	with three colours as each output value depends on at most three input values.
%	A colour vector for $n=5$ could then be $\lbrack c_0, c_1, c_2, c_1, c_0\rbrack$
%	where input values with the
%	same colour can be computed during the same forward-sweep, so we reduce the
%	number of
%	forward calls to the number of colours used. Worth noting is that \texttt{f}'s
%	colour vector
%	\emph{always}  contains 3 colours; independent of $n$ so the number of forward
%	calls
%	is effectively reduced by  $n-3 + 1$ in this case. (Plus one since we need to
%	compute the colour vector)\newline
%	For finding the colour-vector we can perform a forward sweep of the function
%	with some random input and use some underlying data-structure to solve the
%	graph-colouring problem.
%	Alternatively can one let the user provide the color vector.
%
%	\subsubsection{Bundle Adjustment revisited}
%	For Bundle Adjustment we can also use the same approach by seeding the forward
%	sweep
%	with the camera and point data. Using the data it's possible to figure out the
%	dependencies and generate a
%	coloring, which in this case would be a matrix, one for each data-row.
%	In stochastic gradient descent methods, like used Bundle Adjustment, this color
%	vector can provide a speed-up on subsequent
%	computations of the Jacobian.
%	However if the data is only used once, this approach requires that we compute
%	the coloring again and again
%	and hence would not be very efficient.
%
%
%	\subsection{Sparsity w.r.t. Memory}
%	Julia and STAN uses a sparse data structures but to my knowledge not feasible in
%	Futhark?
%	Need more time to look into this.

	\newpage
	\nocite{*}
	\printbibliography


\end{document}
